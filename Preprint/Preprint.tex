\documentclass[]{elsarticle} %review=doublespace preprint=single 5p=2 column
%%% Begin My package additions %%%%%%%%%%%%%%%%%%%
\usepackage[hyphens]{url}

  \journal{BioR\(\chi\)iv} % Sets Journal name


\usepackage{lineno} % add
\providecommand{\tightlist}{%
  \setlength{\itemsep}{0pt}\setlength{\parskip}{0pt}}

\usepackage{graphicx}
\usepackage{booktabs} % book-quality tables
%%%%%%%%%%%%%%%% end my additions to header

\usepackage[T1]{fontenc}
\usepackage{lmodern}
\usepackage{amssymb,amsmath}
\usepackage{ifxetex,ifluatex}
\usepackage{fixltx2e} % provides \textsubscript
% use upquote if available, for straight quotes in verbatim environments
\IfFileExists{upquote.sty}{\usepackage{upquote}}{}
\ifnum 0\ifxetex 1\fi\ifluatex 1\fi=0 % if pdftex
  \usepackage[utf8]{inputenc}
\else % if luatex or xelatex
  \usepackage{fontspec}
  \ifxetex
    \usepackage{xltxtra,xunicode}
  \fi
  \defaultfontfeatures{Mapping=tex-text,Scale=MatchLowercase}
  \newcommand{\euro}{€}
\fi
% use microtype if available
\IfFileExists{microtype.sty}{\usepackage{microtype}}{}
\bibliographystyle{elsarticle-harv}
\ifxetex
  \usepackage[setpagesize=false, % page size defined by xetex
              unicode=false, % unicode breaks when used with xetex
              xetex]{hyperref}
\else
  \usepackage[unicode=true]{hyperref}
\fi
\hypersetup{breaklinks=true,
            bookmarks=true,
            pdfauthor={},
            pdftitle={Technical variations in the Latarjet procedure impact glenohumeral joint stability: A simulation study},
            colorlinks=false,
            urlcolor=blue,
            linkcolor=magenta,
            pdfborder={0 0 0}}
\urlstyle{same}  % don't use monospace font for urls

\setcounter{secnumdepth}{5}
% Pandoc toggle for numbering sections (defaults to be off)

\newlength{\cslhangindent}
\setlength{\cslhangindent}{1.5em}
\newenvironment{cslreferences}%
  {\setlength{\parindent}{0pt}%
  \everypar{\setlength{\hangindent}{\cslhangindent}}\ignorespaces}%
  {\par}

% Pandoc header



\begin{document}
\begin{frontmatter}

  \title{Technical variations in the Latarjet procedure impact
glenohumeral joint stability: A simulation study}
    \author[Centre for Sports Research,Barwon Centre for Orthopaedic
Research and Education (B-CORE)]{Aaron S. Fox\corref{1}}
  
    \author[Barwon Centre for Orthopaedic Research and Education
(B-CORE),School of Medicine,Orthopaedic Department]{Stephen D. Gill}
  
    \author[Centre for Sports Research]{Jason Bonacci}
  
    \author[Barwon Centre for Orthopaedic Research and Education
(B-CORE),School of Medicine,Orthopaedic Department]{Richard S. Page}
  
      \address[Centre for Sports Research]{Centre for Sports Research,
School of Exercise and Nutrition Sciences, Deakin University, Geelong,
Australia}
    \address[Barwon Centre for Orthopaedic Research and Education
(B-CORE)]{Barwon Centre for Orthopaedic Research and Education (B-CORE),
Barwon Health, St John of Jod Hospital and Deakin University, Geelong,
Australia}
    \address[School of Medicine]{School of Medicine, Deakin University,
Geelong, Australia}
    \address[Orthopaedic Department]{Orthopaedic Department, University
Hospital Geelong, Barwon Health, Geelong, Australia}
      \cortext[1]{Corresponding Author: aaron.f@deakin.edu.au}
  
  \begin{abstract}
  Insert abstract\ldots{}
  \end{abstract}
  
 \end{frontmatter}

\hypertarget{introduction}{%
\section{Introduction}\label{introduction}}

Insert text for first paragraph.

Shoulder instability injuries occur with excessive force that translates
the humeral head out of the shoulder joint socket (Thangarajah and
Lambert, 2016). Shoulder instability injuries are a concerning problem
affecting young athletes in overhead collision sports (e.g.~Australian
football; rugby) (Bohu et al., 2015, p. @Orchard2013). Effective
clinical care is vital to avoid recurrent injuries, as well as reduced
shoulder function and joint degradation (Thangarajah and Lambert, 2016).
Surgery is commonly used to address pathology, restore function, and
correct stability (Kavaja et al., 2012). The Latarjet procedure is a
non-anatomic, open shoulder reconstruction surgery involving a bone
block via transfer of the coracoid process to the anterior glenoid
(i.e.~coracoid bone graft) with the attached conjoint tendon (Latarjet,
1954). The Latarjet procedure is commonly used in cases with significant
glenoid bone loss, large humerus compression fractures, or glenoid and
humeral bone defects (Millett et al., 2005) --- and is effective in
combatting recurrent anterior instability injury (Bonacci et al., 2018,
p. @Bessiere2014). Latarjet procedures are emerging as the preferred
option for shoulder stabilisation, especially in contact sport settings
(Millett et al., 2005, p. @Bonazza2017).

\hypertarget{methods}{%
\section{Methods}\label{methods}}

\begin{itemize}
\tightlist
\item
  Use distance to dislocation as our primary metric then we can compare
  how the presence of a bone defect affects the distance relative to the
  intact glenoid (i.e.~as a \%)
\item
  We can then apply this same approach to testing how completing the
  Latarjet procedure changes the distance, still probably keeping this
  in the same relative scale to the intact model (i.e.~does it get back
  to 100\%, or how close to intact does the Latarjet get to).
\item
  We can then look at the data across Latarjet variations
  (i.e.~displacement in the vertical and horizontal directions, graft
  size) in isolation and see what impact this has. From a practical
  perspective, this might tell us where the optimal positioning is to
  maximise distance to dislocation; whether larger grafts equate to
  better outcomes; whether a smaller graft can be used with optimum
  placement; whether you need bigger grafts if you stray away from
  optimum placement etc. Examining the results from the individual
  variations in the context of one another, given they are on the same
  proportional scale, should theoretically help answer these questions.
\item
  It's also possible that the above mentioned aspects may vary with
  different bone defect sizes or types (e.g.~Hill-Sachs included?)
\end{itemize}

\begin{verbatim}
##      speed           dist       
##  Min.   : 4.0   Min.   :  2.00  
##  1st Qu.:12.0   1st Qu.: 26.00  
##  Median :15.0   Median : 36.00  
##  Mean   :15.4   Mean   : 42.98  
##  3rd Qu.:19.0   3rd Qu.: 56.00  
##  Max.   :25.0   Max.   :120.00
\end{verbatim}

\hypertarget{bibliography-styles}{%
\section{Bibliography styles}\label{bibliography-styles}}

There are various bibliography styles available. You can select the
style of your choice in the preamble of this document. These styles are
Elsevier styles based on standard styles like Harvard and Vancouver.
Please use BibTeX~to generate your bibliography and include DOIs
whenever available.

\hypertarget{references}{%
\section*{References}\label{references}}
\addcontentsline{toc}{section}{References}

\hypertarget{refs}{}
\begin{cslreferences}
\leavevmode\hypertarget{ref-Bessiere2014}{}%
Bessière, C., Trojani, C., Carles, M., Mehta, S.S., Boileau, P., 2014.
The Open Latarjet Procedure Is More Reliable in Terms of Shoulder
Stability Than Arthroscopic Bankart Repair. Clinical Orthopaedics and
Related Research 472, 2345--2351.
doi:\href{https://doi.org/10.1007/s11999-014-3550-9}{10.1007/s11999-014-3550-9}

\leavevmode\hypertarget{ref-Bohu2015}{}%
Bohu, Y., Klouche, S., Lefevre, N., Peyrin, J.-C., Dusfour, B., Hager,
J.-P., Ribaut, A., Herman, S., 2015. The epidemiology of 1345 shoulder
dislocations and subluxations in French Rugby Union players: a
five-season prospective study from 2008 to 2013. British Journal of
Sports Medicine 49, 1535--1540.
doi:\href{https://doi.org/10.1136/bjsports-2014-093718}{10.1136/bjsports-2014-093718}

\leavevmode\hypertarget{ref-Bonacci2018}{}%
Bonacci, J., Manson, B., Bowe, S.J., Gill, S., Seward, H., Hoy, G.,
Page, R., 2018. Operative shoulder instability injury management in
Australian Football League players: A case series. Journal of Science
and Medicine in Sport 21, 760--764.
doi:\href{https://doi.org/10.1016/j.jsams.2017.11.011}{10.1016/j.jsams.2017.11.011}

\leavevmode\hypertarget{ref-Bonazza2017}{}%
Bonazza, N.A., Liu, G., Leslie, D.L., Dhawan, A., 2017. Trends in
Surgical Management of Shoulder Instability. Orthopaedic Journal of
Sports Medicine 5, 232596711771247.
doi:\href{https://doi.org/10.1177/2325967117712476}{10.1177/2325967117712476}

\leavevmode\hypertarget{ref-Kavaja2012}{}%
Kavaja, L., Pajarinen, J., Sinisaari, I., Savolainen, V., Björkenheim,
J.-M., Haapamäki, V., Paavola, M., 2012. Arthrosis of glenohumeral joint
after arthroscopic Bankart repair: a long-term follow-up of 13 years.
Journal of Shoulder and Elbow Surgery 21, 350--355.
doi:\href{https://doi.org/10.1016/j.jse.2011.04.023}{10.1016/j.jse.2011.04.023}

\leavevmode\hypertarget{ref-Latarjet1954}{}%
Latarjet, M., 1954. A propos du traitement des luxations récidivante de
l'épaule. Lyon Chir 49, 994--1003.

\leavevmode\hypertarget{ref-Millett2005}{}%
Millett, P.J., Clavert, P., Warner, J.J.P., 2005. Open Operative
Treatment for Anterior Shoulder Instability. The Journal of Bone \&
Joint Surgery 87, 419--432.
doi:\href{https://doi.org/10.2106/JBJS.D.01921}{10.2106/JBJS.D.01921}

\leavevmode\hypertarget{ref-Orchard2013}{}%
Orchard, J.W., Seward, H., Orchard, J.J., 2013. Results of 2 Decades of
Injury Surveillance and Public Release of Data in the Australian
Football League. The American Journal of Sports Medicine 41, 734--741.
doi:\href{https://doi.org/10.1177/0363546513476270}{10.1177/0363546513476270}

\leavevmode\hypertarget{ref-Thangarajah2016}{}%
Thangarajah, T., Lambert, S., 2016. Management of the unstable shoulder.
British Journal of Sports Medicine 50, 440--447.
doi:\href{https://doi.org/10.1136/bjsports-2015-h2537rep}{10.1136/bjsports-2015-h2537rep}
\end{cslreferences}


\end{document}


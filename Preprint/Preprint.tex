\documentclass[]{elsarticle} %review=doublespace preprint=single 5p=2 column
%%% Begin My package additions %%%%%%%%%%%%%%%%%%%
\usepackage[hyphens]{url}

  \journal{An awesome journal} % Sets Journal name


\usepackage{lineno} % add
\providecommand{\tightlist}{%
  \setlength{\itemsep}{0pt}\setlength{\parskip}{0pt}}

\usepackage{graphicx}
\usepackage{booktabs} % book-quality tables
%%%%%%%%%%%%%%%% end my additions to header

\usepackage[T1]{fontenc}
\usepackage{lmodern}
\usepackage{amssymb,amsmath}
\usepackage{ifxetex,ifluatex}
\usepackage{fixltx2e} % provides \textsubscript
% use upquote if available, for straight quotes in verbatim environments
\IfFileExists{upquote.sty}{\usepackage{upquote}}{}
\ifnum 0\ifxetex 1\fi\ifluatex 1\fi=0 % if pdftex
  \usepackage[utf8]{inputenc}
\else % if luatex or xelatex
  \usepackage{fontspec}
  \ifxetex
    \usepackage{xltxtra,xunicode}
  \fi
  \defaultfontfeatures{Mapping=tex-text,Scale=MatchLowercase}
  \newcommand{\euro}{€}
\fi
% use microtype if available
\IfFileExists{microtype.sty}{\usepackage{microtype}}{}
\bibliographystyle{elsarticle-harv}
\ifxetex
  \usepackage[setpagesize=false, % page size defined by xetex
              unicode=false, % unicode breaks when used with xetex
              xetex]{hyperref}
\else
  \usepackage[unicode=true]{hyperref}
\fi
\hypersetup{breaklinks=true,
            bookmarks=true,
            pdfauthor={},
            pdftitle={Short Paper},
            colorlinks=false,
            urlcolor=blue,
            linkcolor=magenta,
            pdfborder={0 0 0}}
\urlstyle{same}  % don't use monospace font for urls

\setcounter{secnumdepth}{5}
% Pandoc toggle for numbering sections (defaults to be off)

\newlength{\cslhangindent}
\setlength{\cslhangindent}{1.5em}
\newenvironment{cslreferences}%
  {\setlength{\parindent}{0pt}%
  \everypar{\setlength{\hangindent}{\cslhangindent}}\ignorespaces}%
  {\par}

% Pandoc header



\begin{document}
\begin{frontmatter}

  \title{Short Paper}
    \author[Centre for Sports Research,Barwon Centre for Orthopaedic
Research and Education (B-CORE)]{Aaron S. Fox\corref{1}}
  
    \author[Barwon Centre for Orthopaedic Research and Education
(B-CORE),School of Medicine,Orthopaedic Department]{Stephen D. Gill}
  
    \author[Centre for Sports Research]{Jason Bonacci}
  
    \author[Barwon Centre for Orthopaedic Research and Education
(B-CORE),School of Medicine,Orthopaedic Department]{Richard S. Page}
  
      \address[Centre for Sports Research]{Centre for Sports Research,
School of Exercise and Nutrition Sciences, Deakin University, Geelong,
Australia}
    \address[Barwon Centre for Orthopaedic Research and Education
(B-CORE)]{Barwon Centre for Orthopaedic Research and Education (B-CORE),
Barwon Health, St John of Jod Hospital and Deakin University, Geelong,
Australia}
    \address[School of Medicine]{School of Medicine, Deakin University,
Geelong, Australia}
    \address[Orthopaedic Department]{Orthopaedic Department, University
Hospital Geelong, Barwon Health, Geelong, Australia}
      \cortext[1]{Corresponding Author: aaron.f@deakin.edu.au}
  
  \begin{abstract}
  This is the abstract.

  It consists of two paragraphs.
  \end{abstract}
  
 \end{frontmatter}

\hypertarget{introduction}{%
\section{Introduction}\label{introduction}}

Insert text\ldots{}

\hypertarget{methods}{%
\section{Methods}\label{methods}}

\begin{itemize}
\tightlist
\item
  Use distance to dislocation as our primary metric then we can compare
  how the presence of a bone defect affects the distance relative to the
  intact glenoid (i.e.~as a \%)
\item
  We can then apply this same approach to testing how completing the
  Latarjet procedure changes the distance, still probably keeping this
  in the same relative scale to the intact model (i.e.~does it get back
  to 100\%, or how close to intact does the Latarjet get to).
\item
  We can then look at the data across Latarjet variations
  (i.e.~displacement in the vertical and horizontal directions, graft
  size) in isolation and see what impact this has. From a practical
  perspective, this might tell us where the optimal positioning is to
  maximise distance to dislocation; whether larger grafts equate to
  better outcomes; whether a smaller graft can be used with optimum
  placement; whether you need bigger grafts if you stray away from
  optimum placement etc. Examining the results from the individual
  variations in the context of one another, given they are on the same
  proportional scale, should theoretically help answer these questions.
\item
  It's also possible that the above mentioned aspects may vary with
  different bone defect sizes or types (e.g.~Hill-Sachs included?)
\end{itemize}

\begin{verbatim}
##      speed           dist       
##  Min.   : 4.0   Min.   :  2.00  
##  1st Qu.:12.0   1st Qu.: 26.00  
##  Median :15.0   Median : 36.00  
##  Mean   :15.4   Mean   : 42.98  
##  3rd Qu.:19.0   3rd Qu.: 56.00  
##  Max.   :25.0   Max.   :120.00
\end{verbatim}

\hypertarget{bibliography-styles}{%
\section{Bibliography styles}\label{bibliography-styles}}

There are various bibliography styles available. You can select the
style of your choice in the preamble of this document. These styles are
Elsevier styles based on standard styles like Harvard and Vancouver.
Please use BibTeX~to generate your bibliography and include DOIs
whenever available.

Here are two sample references: Feynman and Vernon Jr. (1963; Dirac,
1953).

\hypertarget{references}{%
\section*{References}\label{references}}
\addcontentsline{toc}{section}{References}

\hypertarget{refs}{}
\begin{cslreferences}
\leavevmode\hypertarget{ref-Dirac1953888}{}%
Dirac, P.A.M., 1953. The lorentz transformation and absolute time.
Physica 19, 888--896.
doi:\href{https://doi.org/10.1016/S0031-8914(53)80099-6}{10.1016/S0031-8914(53)80099-6}

\leavevmode\hypertarget{ref-Feynman1963118}{}%
Feynman, R.P., Vernon Jr., F.L., 1963. The theory of a general quantum
system interacting with a linear dissipative system. Annals of Physics
24, 118--173.
doi:\href{https://doi.org/10.1016/0003-4916(63)90068-X}{10.1016/0003-4916(63)90068-X}
\end{cslreferences}


\end{document}

